\documentclass[l etterpaper,11pt]{article}

\usepackage[spanish]{babel}
\usepackage[letterpaper,top=2cm,bottom=2cm,left=3cm,right=3cm,marginparwidth=1.75cm]{geometry}
\usepackage{amsmath}
\usepackage{graphicx}
\usepackage[colorlinks=true, allcolors=black]{hyperref}
\usepackage[utf8]{inputenc}


\begin{document}

\begin{titlepage}
\centering

\Large Ecuaciones diferenciales para las que se puede tener soluciones explícitas \par

\vspace{5cm}

\Large Profesor: \par
\Large Jhonatan Collazos Ramírez \par
\vspace{6cm}

\Large Autor: \par
\Large Daniel Fernando Ordoñez Moncayo \par
\vspace{6cm}

{\Large Universidad del Cauca \par}
{\large Facultad de Ingenier\'ia civil \par}
{\large Programa de Ingenier\'ia civil \par}
{\large Ecuaciones diferenciales  \par}
{\large Santander de Quilichao - Cauca \par}
{\Large Agosto de 2022 \par}
\end{titlepage}


\pagebreak

\tableofcontents


\newpage
\section{Introducción}

En carreras como ingenieria tenemos una relacion directa con los numeros,y necesitamos de diverosos metodos matematicos que nos lleven a solucionar problemas que se nos presentan en nuetra vida profesional,para ello en el presente documento se van a exponer algunos de los metodos matematicos para resolver ecuaciones diferenciales para las que se puden obtener soluciones exactas,usaremos metodos como la solucion de ecuaciones diferenciales mediante exctas y factor integrante al igual que por variables separables y pos sustitucion.

\newpage

{\section{Ecuaciones diferenciales exactas}}

Para comprender mejor las ecuaciones diferenciales exactas veamos un teorema que nos permite saber si la ecuación diferencial es exacta o no. Si la ecuación es exacta, entonces tenemos garantizado la existencia de una función $f$ tal que $f\left(x,y\right)=c$, dicha función será la solución de la ecuación exacta.



\subsection{Teorema 1}

 Sean $ M(x,y) $ y $ N(x,y)$ funciones continuas y con primeras derivadas parciales continuas en una región rectangular U definida por $a<x<b $ y $ c<y<d.$ Entonces, una condición necesaria y suficiente para que\\
\begin{center}
$M(x,y)$  y $ N(x,y)$\\
\end{center}


Sea una diferencial exacta es que\\

\begin{flushleft} 
$(1)$
\end{flushleft} 
\begin{center}
$\frac{\partial M}{\partial y}=\frac{\partial N}{\partial x}$ 
\end{center} 



\subsubsection{Demostración}

Supongamos que $ M(x,y)dx+N(x,y)dy $ es exacta, entonces por definición existe alguna función $ f $tal que para toda $ x\ $ en $ U $ se satisface lo siguiente.\\
\begin{center}
$M(x,y)dx+N(x,y)dy=\frac{\partial f}{\partial x}dx+\frac{\partial f}{\partial y}dy$ \\
\end{center}

Esta relación sólo se cumple si\\
\begin{center}
$M\left(x,y\right)=\frac{\partial f}{\partial x}\ \ \ \ \ \ y\ \ \ \ \ N\left(x,y\right)=\frac{\partial f}{\partial y}$\\
\end{center}

Si derivamos parcialmente la expresión\\
\begin{center}

$M(x,y)=\frac{\partial f}{\partial x}$
\end{center}

con respecto a y en ambos lados, obtenemos
\begin{center}

$\frac{\partial M}{\partial y}=\frac{\partial}{\partial y}(\frac{\partial f}{\partial x})=\frac{\partial^2f}{\partial y\partial x}=\frac{\partial^2f}{\partial y\partial x}=\frac{\partial}{dx}(\frac{\partial f}{\partial y})=\frac{\partial N}{\partial x}$
\end{center}

Donde
\begin{center}

$\frac{\partial^2f}{\partial y\partial x}=\frac{\partial^2f}{\partial y\partial x}$\\

\end{center}

Se cumple debido a que las primeras derivadas parciales de $ M(x,y)\ y\ N(x,y) $ son continuas en $ U.$
Si es posible encontrar una función $f$ tal que se cumple (5), entonces la condición
\begin{center}

 $\frac{\partial M}{\partial y}=\frac{\partial N}{\partial x}$\\
\end{center}

Es necesaria y suficiente. Encontrar la función $f $en realidad corresponde a un método de resolución de ecuaciones exactas y lo desarrollaremos a continuación.





\subsubsection{Solución a las ecuaciones exactas}


La ecuación diferencial que queremos resolver es de la forma

La ecuación diferencial que queremos resolver es de la forma
\begin{center}

$M(x,y)dx+N(x,y)dy=0$\\
\end{center}

Por el teorema anterior sabemos que siempre y cuando se cumpla que
\begin{center}

$\frac{\partial M}{\partial y}=\frac{\partial N}{\partial x}$\\
\end{center}

entonces debe existir una función f para la que
\begin{center}

$\frac{\partial f}{\partial x}=M(x,y)\ \ \ y\ \ \ \frac{\partial f}{\partial y}=N(x,y)$\\
\end{center}

Para obtener la función $ f(x,y) $ debemos integrar la primera ecuación con respecto a $ x $ manteniendo a $ y $ constante o integrar la segunda ecuación con respecto a $ y\ $ manteniendo a $ x $ constante, vamos a hacer el primer caso 
Tomando el primer caso, integremos la primera ecuación con respecto a $ x.$
\begin{flushleft} 
$(3)$
\end{flushleft}

\begin{center}

$\int\partial f\partial xdx=\int M(x,y)dx$\\
$f(x,y)=\int M(x,y)dx+g(y)$\\
\end{center}

Hemos hecho uso del teorema fundamental del cálculo y la función g(y) corresponde a la constante de integración, es constante en $ x, $ pero sí puede variar en y ya que en este caso la estamos considerando como una constante al hacer la integral.
Ahora derivemos a (6) con respecto a $ y $. 
\begin{center}

$\frac{\partial f}{\partial y}=\frac{\partial}{\partial y}(\int M(x,y)dx+g(y))=\frac{\partial}{\partial y}(\int M(x,y)dx)+\frac{dg}{dy}$\\
\end{center}

Pero
\begin{center}

$\frac{\partial f}{\partial y}=N(x,y)$\\
\end{center}

Entonces,
\begin{center}

$\frac{\partial}{\partial y}(\int M(x,y)dx)+\frac{dg}{dy}=N(x,y)$\\
\end{center}

Despejemos a
\begin{center}


$\frac{dg}{dy}=g\prime(y)$\\
\end{center}

Se tiene,
\begin{flushleft} 
$(4)$
\end{flushleft}

\begin{center}

$g\prime(y)=N(x,y)-\frac{\partial}{\partial y}(\int M(x,y)dx)$\\
\end{center}

Lo que nos interesa en obtener la función $ f(x,y),\ $ así que podemos integrar la ecuación (4) con respecto a y y sustituir $ g(y) $ en la ecuación (3). Como sabemos, la solución implícita es $ f(x,y)=c.$  Integremos la ecuación (4).
\begin{flushleft} 
$(5)$
\end{flushleft}

\begin{center}

$g(y)=\int N(x,y)dy-\int[\frac{\partial}{\partial y}(\int M(x,y)dx)]dy$\\
\end{center}

Sustituimos el resultado (5) en la ecuación (3) e igualamos el resultado a la constante $ c.$
\begin{center}
\begin{flushleft} 
$(6)$
\end{flushleft}


$f(x,y)=\int M(x,y)dx+\int N(x,y)dy-\int[\frac{\partial}{\partial y}(\int M(x,y)dx)]dy=c$\\
\end{center}

De esta manera habremos encontrado una solución implícita de la ecuación diferencial exacta.
Una observación interesante es que la función $ g\prime(y) $ es independiente de $ x, $ la manera de comprobarlo es con el siguiente resultado.
\begin{center}

$\frac{\partial g}{\partial x}=\frac{\partial}{\partial x}[N(x,y)-\partial\partial y(\int M(x,y)dx)]$\\
$=\frac{\partial N}{\partial x}-\frac{\partial}{\partial x}(\frac{\partial}{\partial y}\int M(x,y)dx)$\\
$=\frac{\partial N}{\partial x}-\frac{\partial}{\partial y}(\frac{\partial}{\partial x}\int M(x,y)dx)$\\
$=\frac{\partial N}{\partial x}-\frac{\partial M}{\partial y}$\\
$=0$\\
\end{center}

Ya que
\begin{center}

$\frac{\partial M}{\partial y}=\frac{\partial N}{\partial x}$\\
\end{center}

Las ecuaciones (3), (5) y (6) son el resultado de tomar el primer caso. Si realizas el segundo caso en el que a la ecuación
\begin{center}

$\frac{\partial f}{\partial y}=N(x,y)$\\
\end{center}

se integra con respecto a $ y $ y al resultado se deriva con respecto a $ x $ obtendremos las expresiones análogas, dichas expresiones son, respectivamente
\begin{flushleft} 
$(7)$
\end{flushleft}

\begin{center}

$f(x,y)=\int N(x,y)dy+h(x)$\\
\end{center}
\begin{flushleft} 
$(8)$
\end{flushleft}

\begin{center}

$h(x)=\int M(x,y)dx-\int[\partial\partial x(\int N(x,y)dy)]dx$\\
\end{center}

\begin{center}

$y$\\
\end{center}

\begin{flushleft} 
$(9)$
\end{flushleft}

\begin{center}


$f(x,y)=\int N(x,y)dy+\int M(x,y)dx-\int[\frac{\partial}{\partial x}(\int N(x,y)dy)]dx=c$\\

\end{center}








\subsection{Método de solución de ecuaciones diferenciales exactas}

Hemos desarrollado la teoría sobre cómo obtener la solución $ f(x,y) $ de las ecuaciones diferenciales exactas. Debido a que no se recomienda memorizar los resultados, presentamos a continuación la siguiente serie de pasos o algoritmo que se recomiendan seguir para resolver una ecuación diferencial exacta.

\begin{enumerate}
\item  El primer paso es verificar que la ecuación diferencial
\begin{center}
$M(x,y)dx+N(x,y)dy=0$\\
\end{center}

sea exacta para garantizar la existencia de la función $\ f  $ tal que $ f(x,y)=c. $ Para verificar este hecho usamos el criterio para una diferencial exacta que consiste en verificar que se cumple la relación
\begin{center}

$\frac{\partial M}{\partial y}=\frac{\partial N}{\partial x}$\\
\end{center}

\item  Una vez que verificamos que la ecuación es exacta tenemos garantizado que existe una función f\ tal que $f(x,y)=c\ $es una solución implícita de la ecuación diferencial. Para determinar dicha función definimos
\begin{center}

$\frac{\partial f}{\partial x}=M(x,y)\ \ \ y\ \ \ \frac{\partial f}{\partial y}=N(x,y)$\\
\end{center}

\item  El siguiente paso es integrar alguna de las ecuaciones anteriores en su respectiva variable, se recomienda integrar la que sea más sencilla de resolver, de esta manera obtendremos
\begin{center}

$f(x,y)=\int M(x,y)dx+g(y)\ \ \ \ o\ \ \ \ f(x,y)=\int N(x,y)dy+h(x)$\\

\end{center}

\item  Después derivamos parcialmente a la función $ f(x,y)\ $ con respecto a la variable y o x según la elección hecha en el paso anterior, de manera que obtendremos el resultado
\begin{center}

$\frac{\partial f}{\partial y}=\frac{\partial}{\partial y}(\int M(x,y)dx)+\frac{dg}{dy}=N(x,y)$\\
\end{center}

\begin{flushleft}
O bien,\\
 \end{flushleft}
\begin{center}
$\frac{\partial f}{\partial x}=\frac{\partial}{\partial x}(\int N(x,y)dy)+\frac{dh}{dx}=M(x,y)$\\
\end{center}

\item De los resultados anteriores obtendremos una expresión para $\frac{dg}{dy}, $ o $ para \frac{dh}{dx}, $ debemos integrar estas expresiones para obtener las funciones $ g(y) $ o $ h(x).$
\item El último paso es sustituir las funciones $ g(y) $ o $ h(x) $ en la ecuación $ f(x,y)=c  $ lo que nos devolverá, en general, una solución implícita de la ecuación diferencial exacta.
\end{enumerate}



\subsubsection{Ejemplo}

Vamos a resolver la siguiente ecuación diferencial
\begin{center}

$({4x}^3-{4xy}^2+y)dx+({4y}^3-{4x}^2y+x)dy=0$\\

\end{center}

Solución:\\
La ecuación diferencial es de la forma $ M(x,y)dx+N(x,y)dy=0 ,$ de manera que podemos definir
\begin{center}
$M(x,y)={4x}^3-{4xy}^2+y\ \ \ \ \ y\ \ \ \ N(x,y)={4y}^3-{4x}^2y+x$\\
\end{center}

Ambas funciones son continuas y tienen derivadas parciales continuas en cualquier región$  U\in R2,  $ entonces podemos aplicar el criterio para una diferencial exacta. Verifiquemos que se satisface la relación (1).
\begin{center}

$\frac{\partial M}{\partial y}=-8xy+1\ \ \ \ \ y\ \ \ \ \ \frac{\partial N}{\partial x}=-8xy+1$\\
\end{center}

En efecto,
\begin{center}
$\frac{\partial M}{\partial y}=\frac{\partial N}{\partial x}$\\
\end{center}

Por lo tanto, la ecuación diferencial sí es exacta, esto nos garantiza la existencia de una función $ f $tal que $ f(x,y)=c $ es solución, entonces podemos definir
\begin{center}

$\frac{\partial f}{\partial x}=M(x,y)={4x}^3-{4xy}^2+y\ \ \ \ \ \ y\ \ \ \ \ \ \frac{\partial f}{\partial y}=N(x,y)={4y}^3-{4x}^2y+x$\\

\end{center}

El tercer paso nos indica que debemos integrar una de las ecuaciones anteriores, en este caso elegiremos integrar la ecuación
\begin{center}

$\frac{\partial f}{\partial x}={4x}^3-{4xy}^2+y$\\
\end{center}

con respecto a la variable $ x.$
\begin{center}

$\int\frac{\partial f}{\partial x}dx=\int({4x}^3-{4xy}^2+y)dx$\\
\end{center}

Del lado izquierdo aplicamos el teorema fundamental del cálculo y del lado derecho resolvemos la integrar, el resultado es
\begin{center}

$f(x,y)=x^4-{2x}^2{{2x}^2y}^2+xy+g(y)$\\
\end{center}

La función $ g(y) $ es la constante que considera a todas las constantes que aparecen al integrar y decimos que es constante porque no depende de la variable $ x, $ pero es posible que pueda depender de la variable $y$.
El cuarto paso es derivar la última ecuación con respecto a la variable $ y $ ya que deseamos conocer a $\frac{dg}{dy}=g\prime(y).$
\begin{center}

$\frac{\partial f}{\partial y}=-{4x}^2y+x+\frac{dg}{dy}$\\
\end{center}

Y sabíamos que
\begin{center}

$\frac{\partial f}{\partial y}={4y}^3-{4x}^2y+x$\\
\end{center}

Igualando ambas ecuaciones, obtenemos
\begin{center}

${-4x}^2y+x+\frac{dg}{dy}={4y}^3{-4x}^2y+x$\\
\end{center}

Para que esta igualdad se cumpla es necesario que
\begin{center}

$\frac{dg}{dy}={4y}^3$\\
\end{center}

Ahora que ya conocemos a $  \frac{dg}{dy}=g\prime(y), $ la integramos con respecto a $y$. Esto corresponde al penúltimo paso.
\begin{center}

$\int\frac{dg}{dy}dy=\int{4y}^3dy$\\
$g(y)=y4$\\
\end{center}

El último paso es sustituir el resultado $ g(y) $ en la función $\ f(x,y)=c.$  En la integración anterior omitimos a las constantes porque podemos englobarlas en la constante $c.$4
\begin{center}

$f(x,y)=x^4-{2x}^2y^2+xy+y^4=c$\\
\end{center}

de donde
\begin{center}

${(x^2-y^2)}^2+xy=c$\\
\end{center}

Por lo tanto, la solución (implícita) de la ecuación diferencial exacta
\begin{center}

$({4x}^3-{4xy}^2+y)dx+({4y}^3-{4x}^2y+x)dy=0$\\
\end{center}

Es
\begin{center}

${(x^2-y^2)}^2+xy=c$\\
\end{center}


Por su puesto que hay ecuaciones diferenciales de la forma $ M(x,y)dx+N(x,y)dy=0 ,$ que no cumplen con la condición (1), es decir, que no son exactas, en estos casos es posible apoyarnos de una función auxiliar tal que si multiplicamos a la ecuación diferencial por esta función se volverá exacta, si esto ocurre a dicha función la llamamos factor integrante. Así es, usaremos un método similar al método por factor integrante de las ecuaciones lineales, pero esta vez es para convertir a una ecuación diferencial no exacta en exacta.


{\section{Factor integrante }}

La finalidad de multiplicar por un factor integrante será de ayudará a trabajar con ecuaciones diferenciales de la forma
\begin{center}

$M(x,y)dx+N(x,y)dy=0$\\
\end{center}

que no son exactas. Lo que se espera es que multiplicando por un factor integrante $ \mu(x,y) $  a la ecuación no exacta ésta se vuelva una ecuación exacta.
Consideremos la ecuación
\begin{center}

$M(x,y)dx+N(x,y)dy=0$\\
\end{center}

pero que no es exacta, esto significa que el lado izquierdo de la ecuación no corresponde a la diferencial de alguna función $ f(x,y). $ Supongamos que existe una función $ \mu(x,y) $ tal que al multiplicar la ecuación diferencial por esta función se vuelve una ecuación diferencial exacta. Es decir, la ecuación
\begin{flushleft} 
$(10)$
\end{flushleft}

\begin{center}

$\mu(x,y)M(x,y)dx+\mu(x,y)N(x,y)dy=0$\\
\end{center}

ahora es exacta y se puede resolver con el método que ya conocemos. Lo que veremos ahora es un método para determinar el factor integrante $ \mu(x,y). $
Por el criterio de diferencial exacta, la ecuación diferencial (10) es exacta si
\begin{flushleft} 
$(11)$
\end{flushleft}

\begin{center}

$\frac{\partial(\mu M)}{\partial y}=\frac{\partial(\mu N)}{\partial x}$\\
\end{center}

Usando la regla del producto, la ecuación anterior se puede escribir como
\begin{center}

$\mu\frac{\partial M}{\partial y}+\frac{\partial\mu}{\partial y}M=\mu\frac{\partial N}{\partial x}+\frac{\partial\mu}{\partial x}N$\\
\end{center}

Reordenando los términos obtenemos la siguiente expresión.
\begin{flushleft} 
$(12)$
\end{flushleft}

\begin{center}

$\frac{\partial\mu}{\partial x}N-\frac{\partial\mu}{\partial y}M=(\frac{\partial M}{\partial y}-\frac{\partial N}{\partial x})\mu$\\
\end{center}


Para determinar la función $\mu\left(x,y\right)$ debemos resolver esta ecuación diferencial parcial, sin embargo, no estamos en condiciones de hacerlo, pues no sabemos resolver ecuaciones diferenciales parciales. Para simplificar el problema vamos a considerar la hipótesis de que la función $  \mu\left(x,y\right) $ depende sólo de una variable, consideremos por ejemplo que $\mu$ depende sólo de $ x$, así se cumple que
\begin{center}

$\frac{\partial\mu}{\partial x}=\frac{d\mu}{dx}\ \ \ \ y\ \ \ \ \frac{\partial\mu}{\partial y}=0$\\
\end{center}

Con estas hipótesis la ecuación (12) se puede escribir de la siguiente forma.
\begin{flushleft} 
$(13)$
\end{flushleft}

\begin{center}

$\frac{d\mu}{dx}=\frac{1}{N}(\frac{\partial M}{\partial y}-\frac{\partial N}{\partial x})\mu$\\
\end{center}

Seguimos en problemas si el cociente de la derecha depende tanto de $ x$ como de y. En el caso en el que dicho cociente sólo depende de $ x, $entonces la ecuación será separable, así como lineal.
Supongamos que la ecuación (14) sólo depende de la variable $ x $, entonces dividimos toda la ecuación por $ \mu $ para separar las variables.
\begin{center}

$\frac{1}{\mu}\frac{d\mu}{dx}=\frac{1}{N}(\frac{\partial M}{\partial y}-\frac{\partial N}{\partial x})$\\
\end{center}

Integremos ambos lados de la ecuación con respecto a la variable $ x $.

\begin{center}

$\int{\frac{1}{\mu}\frac{d\mu}{dx}dx}=\int{\frac{1}{N}\left(\frac{\partial M}{\partial y}-\frac{\partial N}{\partial x}\right)dx}$\\
$\ln{\left|\mu\left(x\right)\right|}=\int{\frac{1}{N}\left(\frac{\partial M}{\partial y}-\frac{\partial N}{\partial x}\right)dx}$\\
\end{center}

Finalmente apliquemos la exponencial en ambos lados de la ecuación.
\begin{flushleft} 
$(14)$
\end{flushleft}

\begin{center}

$\mu(x)=e^{\int\frac{1}{N}(\frac{\partial M}{\partial y}-\frac{\partial N}{\partial x})dx}$\\
\end{center}

Es totalmente análogo el caso en el que el factor integrante es sólo función de la variable $ y $, en este caso se cumple
\begin{center}

$\frac{\partial\mu}{\partial x}=0\ \ \ \ \ y\ \ \ \ \frac{\ \partial\mu}{\ \partial y}=\frac{d\mu}{dy}$\\
\end{center}

Es así que la ecuación (12) queda de la siguiente forma.
\begin{flushleft} 
$(15)$
\end{flushleft}
\begin{center}

$\frac{d\mu}{dy}=\frac{1}{M}(\frac{\partial N}{\partial x}-\frac{\partial M}{\partial y})\mu$\\
\end{center}

Si el cociente de la derecha sólo depende de la variable $ y $, entonces se puede resolver la ecuación (15), obteniendo
\begin{flushleft} 
$(16)$
\end{flushleft}

\begin{center}

$\mu(y)=e^{\int\frac{1}{M}(\frac{\partial N}{\partial x}-\frac{\partial M}{\partial y})dy}$\\
\end{center}

Las funciones (14) y (16) corresponden a la forma del factor integrante que vuelven a la ecuación no exacta en exacta, según las condiciones que se presenten.
A manera de resumen, para el caso en el que la ecuación diferencial
\begin{center}

$M(x,y)dx+N(x,y)dy=0$\\
\end{center}


No es exacta probamos los siguientes dos casos:	
\begin{enumerate}

\item Si
\begin{center}

$\frac{1}{N}(\frac{\partial M}{\partial y}-\frac{\partial N}{\partial x})$\\
\end{center}

es una función sólo de $ x, $ entonces un factor integrante para la ecuación (10) es:

\begin{center}

$\mu\left(x\right)=e^{\int{\frac{1}{N}\left(\frac{\partial M}{\partial y}-\frac{\partial N}{\partial x}\right)dx}}$\\
\end{center}

\item Si
\begin{center}
$\frac{1}{M}(\frac{\partial N}{\partial x}-\frac{\partial M}{\partial y})$\\
\end{center}

es una función sólo de y, entonces un factor integrante para la ecuación (10) es:
\begin{center}

$\mu\left(y\right)=e^{\int{\frac{1}{M}\left(\frac{\partial N}{\partial x}-\frac{\partial M}{\partial y}\right)dy}}$\\
\end{center}

\end{enumerate}





\subsection{Ejemplo}


Vamos a resolver la siguiente ecuación diferencial no exacta.
\begin{center}

$(1-\frac{y}{x}e^{y/x})dx+e^{y/x}dy=0$\\
\end{center}

Solución: Verifiquemos que no es una ecuación exacta, definamos
\begin{center}

$M\left(x,y\right)=1-\frac{y}{x}e^{y/x}\ \ \ \ \ \ y\ \ \ \ \ N\left(x,y\right)=e^{y/x}$\\
\end{center}

Calculemos las derivadas parciales correspondientes.
\begin{center}

$\frac{\partial M}{\partial y}=-\frac{1}{x}e^\frac{y}{x}-\frac{y}{x^2}e^\frac{y}{x}\ \ \ \ \ y\ \ \ \ \ \frac{\partial N}{\partial x}=-\frac{y}{x^2}e^\frac{y}{x}$\\

\end{center}

Como
\begin{center}

$\frac{\partial M}{\partial y}\neq\frac{\partial N}{\partial x}$\\
\end{center}

entonces la ecuación diferencial no es exacta. Para hacerla exacta debemos encontrar un factor integrante que dependa de $ x $ o de  $ y $ , para ello primero debemos ver si el cociente
\begin{center}


$\frac{1}{N}(\frac{\partial M}{\partial y}-\frac{\partial N}{\partial x}) $\\
\end{center}

es una función sólo de $ x $ o si el cociente
\begin{center}

$\frac{1}{M}(\frac{\partial N}{\partial x}-\frac{\partial M}{\partial y}) $\\
\end{center}

es una función sólo de $ y $. Calculemos ambos cocientes usando los resultados anteriores.
\begin{center}

$\frac{1}{M}\left(\frac{\partial N}{\partial x}-\frac{\partial M}{\partial y}\right)=\left(1-\frac{y}{x}e^{y/x}\right)^{-1}\left(-\frac{y}{x^2}e^{y/x}+\frac{1}{x}e^\frac{y}{x}+\frac{y}{x^2}e^{y/x}\right)=\frac{\frac{1}{x}e^{y/x}}{1-\frac{y}{x}e^{y/x}}$\\
\end{center}

Y
\begin{center}

$\frac{1}{N}\left(\frac{\partial M}{\partial y}-\frac{\partial N}{\partial x}\right)=e^{-y/x}\left(-\frac{1}{x}e^{y/x}-\frac{y}{x^2}e^{y/x}+\frac{y}{x^2}e^{y/x}\right)=-\frac{1}{x}$\\
\end{center}


Este último cociente es el que nos sirve ya que sólo depende de la variable$ x.$ Calculemos el factor integrante, en este caso corresponde a la expresión (14).
\begin{center}

$\mu\left(x\right)=e^{\left[\int{\frac{1}{N}\left(\frac{\partial M}{\partial y}-\frac{\partial N}{\partial x}\right)}\right]}$\\
$=e^{\left[\int{\frac{1}{x}dx}\right]}$\\

$=-e^{\ln{\left|x\right|}}$\\
$=x^{-1}$\\
\end{center}

Por lo tanto, el factor integrante es
\begin{center}

$\mu(x)=\frac{1}{x}$\\
\end{center}

Multipliquemos ambos lados de la ecuación original por el factor integrante.
\begin{center}

$\frac{1}{x}(1-\frac{y}{x}e^{y/x})dx+\frac{1}{x}e^{y/x}dy=0$\\
$(\frac{1}{x}-\frac{y}{x^2}e^{y/x})dx+\frac{1}{x}e^{y/x}dy=0$\\
\end{center}

Verifiquemos que la última expresión corresponde a una ecuación diferencial exacta. Definamos
\begin{center}

$M\prime(x,y)=\mu(x)M(x,y)\ \ \ \ \ y\ \ \ \ \ N\prime(x,y)=\mu(x)N(x,y) $\\
\end{center}

Entonces,
\begin{center}

$M\prime(x,y)=\frac{1}{x}-\frac{y}{x^2}e^{y/x}\ \ \ \ \ y\ \ \ \ \ N^(x,y)=\frac{1}{x}e^{y/x}$\\
\end{center}

Calculemos las derivadas parciales correspondientes.
\begin{center}

$\frac{\partial M\prime}{\partial y}=-\frac{1}{x^2}e^{y/x}-\frac{y}{x^3}e^{y/x}\ \ \ \ \ y\ \ \ \ \ \frac{\partial N\prime}{\partial x}=-\frac{1}{x^2}e^{y/x}-\frac{y}{x^3}e^{y/x}$\\
\end{center}

En efecto,
\begin{center}

$\frac{\partial M\prime}{\partial y}=\frac{\partial N\prime}{\partial x}$\\
\end{center}

La nueva ecuación sí es exacta, esto nos garantiza que existe una función f tal que $ f(x,y)=c $ es solución de la ecuación exacta, dicha función debe satisfacer que
\begin{center}

$\frac{\partial f}{\partial x}=M\prime(x,y)=\frac{1}{x}-\frac{y}{x^2}e^{y/x}\ \ \ \ \ y\ \ \ \ \frac{\ \partial f}{\ \partial y}=N^(x,y)=\frac{1}{x}e^{y/x}$\\
\end{center}

Es nuestra elección que ecuación integrar, sin embargo, notamos que la función $ N’(x,y) $ es la más sencilla de integrar, así que integremos esta ecuación con respecto a $ y $.
\begin{center}

$\int\frac{\ \partial f}{\ \partial y}dy=\int\frac{1}{x}e^{y/x}dy$\\
$f(x,y)=e^{y/x}+h(x) $\\
\end{center}

Derivemos parcialmente este resultado con respecto a la variable $ x$.
\begin{center}

$\frac{\partial f}{\partial x}=-\frac{y}{x^2}e^{y/x}+\frac{dh}{dx}$\\
\end{center}

Pero sabemos que
\begin{center}

$\frac{\partial f}{\partial x}=M\prime(x,y)=\frac{1}{x}-\frac{y}{x^2}e^{y/x}$\\
\end{center}

Igualemos ambas ecuaciones.
\begin{center}

$\frac{1}{x}-\frac{y}{x^2}e^{y/x}=-\frac{y}{x^2}e^{y/x}+\frac{dh}{dx}$\\
\end{center}

Para que se cumpla esta igualdad es necesario que
\begin{center}

$\frac{dh}{dx}=\frac{1}{x}$\\
\end{center}

Integremos esta ecuación con respecto a $ x $ omitiendo las constantes.
\begin{center}

$\int\frac{dh}{dx}dx=\int\frac{1}{x}dx
$h(x)=$\ln{\left|x\right|}$\\
\end{center}

Sustituimos la función $h(x)$ en la función $ f(x,y) $ e igualamos a una constante $ c $.
\begin{center}

$f\left(x,y\right)=e^{y/x}+\ln{\left|x\right|}=c$\\
\end{center}

Apliquemos la función exponencial
\begin{center}

$e^{{(e}^{y/x}+\ln{(x)})}=e^c$\\
$e^{e^{y/x}}e^{\ln{(x)}}=k$\\
$e^{e^{y/x}}x=k$\\
\end{center}

Donde $ k=ec. $ Por lo tanto, la solución a la ecuación diferencial
\begin{center}

$(1-\frac{y}{x}e^{y/x})dx+e^{y/x}dy=0$\\
\end{center}

Es
\begin{center}

$xe^{e^{y/x}}=k$\\
\end{center}


Aquí concluimos nuestro estudio sobre las ecuaciones diferenciales exactas.


\section{Ecuaciones diferenciales separables}

Definición: Una ecuación diferencial de primer orden de la forma
\begin{equation}
\frac{dy}{dx}=H(x,y)\\
\end{equation}

se dice que es separable o que tiene variables separables siempre que $ H(x,y)$  puede escribirse como el producto de una función de $ x $ y una función de $ y $
\begin{equation}
H(x,y)=g(x)h(y)
\end{equation}

Inmediatamente nos damos cuenta que es una ecuación diferencial no lineal debido a que aparece una función dependiente de la variable dependiente $ y.$
Veamos cómo encontrar la solución general de este tipo de ecuaciones.
Solución a ecuaciones separables
Es conveniente definir la función
\begin{equation}
h(y)=\frac{1}{f(y)}
\end{equation}

de tal manera que la ecuación (1) se pueda escribir de la siguiente forma.
\begin{equation}
\frac{dy}{dx}=\frac{g(x)}{f(y)}
\end{equation}

Esta ecuación la podemos reescribir como
\begin{equation}
f(y)\frac{dy}{dx}=g(x)
\end{equation}

Notemos que en el lado derecho de la igualdad tenemos la función que depende de la variable independiente $ x $, mientras que en el lado izquierdo tenemos la función que depende de la variable dependiente $ y, $ en esta situación decimos que hemos separado a la ecuación diferencial.
Es común encontrar en la literatura que la ecuación (5) se escribe como
\begin{equation}
g(x)dx=f(y)dy\\
\end{equation}

Esta es la forma diferencial de la ecuación (4), es una notación informal, pero nos permite visualizar que hemos sido capaz de separar a las variables, el lado izquierdo sólo depende de $x$ y el lado derecho sólo depende de $ y $.
Podemos integrar ambos lados de la ecuación. Si consideramos la ecuación en la forma (5), entonces integramos ambos lados con respecto a la variable $x$ y si consideramos la ecuación en la forma (6) integramos con respecto a la variable correspondiente.
\begin {center}

$\int f(y)dydxdx=\int g(x)dx$\\
$\int f(y)dy=\int g(x)dx$\\
\end {center}

Sólo es necesario que las antiderivadas
\begin{equation}
F(y)=\int f(y)dy
\end{equation}

Y
\begin{equation}
G(x)=\int g(x)dx
\end{equation}

existan y puedan resolverse. Una vez resueltas las integrales obtendremos una familia uniparamétrica de soluciones que usualmente se expresa de forma implícita.



\subsubsection{Método de separación de variables}

De acuerdo a lo anterior, el algoritmo que se recomienda seguir para resolver ecuaciones diferenciales separables es el siguiente.
\begin{enumerate}
\item  Dada una ecuación diferencial no lineal de primer orden, el primer paso es identificar si es posible que podamos determinar una función $g=g(x)$ que sólo depende de la variable independiente x y una función $f=f(y)$ que sólo depende de la variable dependiente $y$, si esto es posible escribimos a la ecuación diferencial en la siguiente forma.
\begin {center}
$f(y)\frac{dy}{dx}=g(x)$\\
\end {center}

\item  El segundo paso es integrar ambos lados de la ecuación con respecto a la variable $x$. En este caso debemos considerar en todo momento las constantes de integración.
\item  Al resolver la integral $\int f(y)dy$ obtendremos la solución $y(x)$ que estamos buscando, ya sea de forma implícita o explicita, ambas formas son válidas.
\end{enumerate}

Realicemos un ejemplo en el que apliquemos este método.

\subsubsection{Ejemplo}

De acuerdo a lo anterior, el algoritmo que se recomienda seguir para resolver ecuaciones diferenciales separables es el siguiente.
\begin{enumerate}
\item  Dada una ecuación diferencial no lineal de primer orden, el primer paso es identificar si es posible que podamos determinar una función $g=g(x)$ que sólo depende de la variable independiente x y una función $f=f(y)$ que sólo depende de la variable dependiente $y$, si esto es posible escribimos a la ecuación diferencial en la siguiente forma.
\begin {center}
$f(y)\frac{dy}{dx}=g(x)$\\
\end {center}

\item  El segundo paso es integrar ambos lados de la ecuación con respecto a la variable $x$. En este caso debemos considerar en todo momento las constantes de integración.
\item  Al resolver la integral $\int f(y)dy$ obtendremos la solución $y(x)$ que estamos buscando, ya sea de forma implícita o explicita, ambas formas son válidas.
\end{enumerate}

Realicemos un ejemplo en el que apliquemos este método.
Ejemplo: Resolver la ecuación diferencial
\begin{center}
$\frac{dy}{dx}e^{(y-x)}=x$\\
\end{center}

con la condición inicial $ y(0)=ln{\left(2\right)}.$
Solución: El primer paso es determinar si la ecuación es separable, es decir, si podemos hallar las funciones $  g(x)$ y $f(y)$. Vemos que

\begin{center}
$\frac{dy}{dx}e^{(y-x)}=x$\\
$\frac{dy}{dx}e^ye^{-x}=x$\\
$e^y\frac{dy}{dx}=xe^x$\\
\end{center}

Ya logramos escribir a la ecuación en la forma (5), de donde podemos establecer que
\begin{center}
$g(x)=xe^x\ \ \ \ \ y\ \ \ \ f(y)=e^y$\\
\end{center}

Usando la notación diferencial podemos escribir a la ecuación como
\begin {center}
$eydy=xe^xdx$\\
\end {center}

Integremos ambos lados de la ecuación ante la respectiva variable.
\begin{center}
$\int e^ydy=\int xe^xdx$\\
\end{center}

Por un lado,
\begin {center}
$\int e^ydy=e^y+k1$\\
\end {center}

Por otro lado, para la integral de x usemos integración por partes considerando $ u(x)=x\ \ \ y\ \ \ dv(x)=e^x$.
\begin {center}

$\int x e^xdx=xe^x-\int e^xdx$\\
$=xe^x-(e^x+k2)$\\
$=xe^x-e^x-k2$\\
\end {center}

Igualando ambos resultados obtenemos lo siguiente.
\begin {center}
$e^y+k1=xe^x-e^x-k2$\\
$e^y=xe^x-e^x-k2-k1$\\
$e^y=xe^x-e^x+c$\\
\end {center}

En donde  $c=-k2-k1$. Por lo tanto, la solución implícita es
\begin {center}
$e^y=xe^x-e^x+c$\\
\end {center}

Para conocer la solución explícita sólo tomamos el logaritmo natural.
\begin {center}
$y(x)=\ln{\left|xe^x-e^x+c\right|}$\\
\end {center}

Obtengamos la solución particular aplicando la condición inicial $y(0)=\ln{(2)}$\\
\begin {center}
$y\left(0\right)=\ln{\left|0e^0-e^o+c\right|=\ln{(2)}}$\\
$y\left(0\right)=\ln{\left|0-1+c\right|=\ln{(2)}}$\\
\end {center}

De donde,
\begin {center}
$\ln{\left|c-1\right|}=\ln{(2)}$\\
\end {center}

Aplicando la exponencial en ambos lados, se tiene
\begin {center}
$c-1=2$\\
\end {center}

De donde $c=3$. Por lo tanto, la solución particular es
\begin {center}
$e^y=xe^x-e^x+3$\\
\end {center}

O bien,
\begin {center}
$y(x)=\left|xe^x-e^x+3\right|$\\
\end {center}

Este tipo de ecuaciones son muy sencillas de resolver, prácticamente se resuelven aplicando una integración directa.
Veamos ahora las ecuaciones diferenciales no lineales homogéneas, lo interesante de este tipo de ecuaciones es que si hacemos un cambio de variable adecuado las podremos reducir a una ecuación separable las cuales ya sabemos resolver.



\subsection{Ecuaciones diferenciales por sustitución}

Para resolver una ecuación diferencial 
Se reconoce ella cierto tipo de ecuación separable, por ejemplo 
Aplacimos un procedimiento formado por etapas de tipo de ecuación que nos conducen a una función diferenciable, que satisface la ecuación 
A menudo el primer paso es transfórmala en otra ecuación diferencial mediante sustitución 
Supongamos que se quiere transformar la ecuación de primer orden 
\begin{center}
$\frac{dy}{dx}=f(x,y)$\\
\end{center}

Con la sustitución 
\begin{center}

$y=g(x,u)$\\
\end{center}

en que $ u$ se considera función de la variable $x$,si $g$ tiene primeras derivadas parciales entonces por la regla de la cadena da:
\begin{center}

$\frac{dy}{dx}=g_x(x,u)+g_u(x,u)\frac{du}{dx}$\\
\end{center}

Luego al sustituir 
\begin{center}

$\frac{dy}{dx}\ \ \ \ \ con\ \ \ \ \ f(x,y)$\\
\end{center}

En la derivada anterior obtenemos la nueva ecuación diferencial de primer orden 
\begin{center}


$f(x,g(x,u))=g_x(x,u)+g_u(x,u)\frac{du}{dx}$\\
\end{center}



\subsubsection{Metodología de 4 pasos para resolver ecuaciones diferenciales reducidas a variables separables}

\begin{enumerate}



\item Buscamos una sustitución que nos permita transformar a lineal o separable la ED.  Generalmente cuando tenemos:

\begin{equation}
\frac{dy}{dx}=f(Ax + By + C)\\
\end{equation}

Utilizamnos la sustitución:  $u=Ax + By + C$\\

 Despejamos de la nueva función v , la variable y :
 \begin{equation}
y = \frac{u+Ax+C}{B}\\
 \end{equation}
Y derivamos para obtener\\
\begin{equation}
\frac{d y}{d x} = \frac{d u}{B d x} + \frac{A}{B}\\
\end{equation}

\item Sustituimos (10) y (11) en (9):

\begin{eqnarray*}
\frac{d y}{d x} & = & f (A x + B y + C)\\
\frac{d u}{B d x} + \frac{A}{B} & = & f (u)\\
\frac{d u}{B d x} & = & f (u) – \frac{A}{B}\\
\frac{d u}{B d x} & = & \frac{f (u) \ast B – A}{B}
\end{eqnarray*}

\item Separamos variables e integramos:
\begin{eqnarray*}
\frac{d u}{B d x} & = & \frac{f (u) \ast B – A}{B}\\
\frac{d u}{f (u) \ast B – A} & = & dx\\
\int \frac{d u}{f (u) \ast B – A} & = & \int dx + C 1
\end{eqnarray*}

\item Regresamos a las variables originales.


\end{enumerate}






\subsubsection{Ejemplo}

Ejercicios 2.5 Libro Dennis G. Zill (Problema 23)

\begin{equation}
\frac{d y}{d x} = (x + y + 1)^2\\	
\end{equation}
\begin{enumerate}

\item Buscamos una sustitición para transformar en separable la ED.


\begin{center}
Si  $u = x + y + 1$\\
\end{center}
Entonces, despejando $y$, tenemos:
\begin{equation}
y = u – x – 1
\end{equation}
Esto implica:
\begin{equation}
\frac{d y}{d x} = \frac{d u}{d x} – 1
\end{equation}

\item  Sustituimos (13) y (14) en (12):
\begin{eqnarray*}
\frac{d y}{d x} & = & (x + y + 1)^2\\
\frac{d u}{d x} – 1 & = & u^2\\
\frac{d u}{d x} & = & u^2 + 1
\end{eqnarray*}

\item Separamos variables e integramos:
\begin{eqnarray*}
\frac{d u}{d x} & = & u^2 + 1\\
\frac{d u}{u^2 + 1} & = & d x
\end{eqnarray*}

De modo que:

\begin{center}
$\int \frac{d u}{u^2 + 1} = \int d x + C$\\
\end{center}

Integrando, utilizamos la formula:
\begin{center}
$\int \frac{d T}{T^2 + 1} = \arctan (T)$\\
\end{center}

por tanto:
\begin{eqnarray*}
\int \frac{d u}{u^2 + 1} & = & \int d x + C\\
\arctan (u) & = & x + C
\end{eqnarray*}

\item. Regrasamos a las variables originales:
si $u=x+y+1$ , entonces:
\begin{eqnarray*}
\arctan (x + y + 1) & = & x + C\\
x + y + 1 & = & \tan (x + C)\\
y & = & \tan (x + C) – x – 1
\end{eqnarray*}
De modo que la solución es:
\begin{center}
$\ y = \tan{(x + C)} – x -1$\\
\end{center}

\end{enumerate}

\pagebreak
\section{Referencias}

Martínez I Sandoval, 2021,Ecuaciones Diferenciales I: Ecuaciones diferenciales exactas tomado de: https://blog.nekomath.com/about/\\

Martínez L Sandoval, 2021, Ecuaciones diferenciales no lineales de primer orden – Ecuaciones separables y homogéneas tomado de: https://blog.nekomath.com/tag/variables-separables/\\

Vivas M Riverol Mayo 2, 2016, Ecuaciones diferenciales por sustitucion, tomado de: https://ecuaciondiferencialejerciciosresueltos.com/ecuaciones-diferenciales-por-sustitucion/\\



\end{document}





























